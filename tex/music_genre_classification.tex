\documentclass{article}
\usepackage[utf8]{inputenc}
\usepackage[serbian]{babel}

\title{Klasifikacija muzičkih žanrova}
\author{Lucija Miličić \and Natalija Asanović}
\date{Avgust 2022.}

\begin{document}

\maketitle

\newpage

\section{Uvod}

Oznake muzičkih žanrova su korisne za organizovanje pesama, albuma i izvođača u šire grupe koje dele slične muzičke karakteristike. Kako količina muzike koja se izdaje na dnevnoj bazi sve više raste, raste i potreba za efikasnim određivanjem muzičkih žanrova. Cilj automatizacije muzičke klasifikacije je da izbor pesama bude brz i manje težak. Zbog toga klasifikacija muzičkih žanrova predstavlja jedan od važnih zadataka za mnoge aplikacije.

U nastavku rada pokazaćemo kako se klasifikuju muzički žanrovi korišćenjem konvolutivne neuronske mreze.

\subsection{Klasifikacija}
Zadatak klasifikacije jeste da odredi funkciju (klasifilacioni model) koja preslikava skup ulaznih atributa $X = (x^1, x^2, ..., x^k)$ u jednu od predefinisanih vrednosti $y$, gde je $y$ oznaka klase. 

Cilj formiranja modela je primena na materijal kod kojeg je vrednost atributa $y$ nepoznata, radi što preciznijeg predviđanja vrednosti $y$.

Ulazni podaci se obično dele u dva disjunktna skupa:
\begin{itemize}
    \item skup za trening - koristi se za formiranje modela
    \item skup za testiranje - koristi se za proveru ispravnosti modela
\end{itemize}

Moguće je uvesti i treći skup, skup za validaciju. On se koristi u toku formiranja modela, kako bi se izbegla preterana prilagođenost modela podacima.

\subsection{Tehnike klasifikacije}
Osnovne tehnike / metode klasifikacije su:
\begin{itemize}
    \item drveta odlučivanja
    \item neuronske mreze
    \item statistički zasnovane metode
    \item određivanje najbližeg suseda
    \item mašine sa potpornim vektorima
    \item ...
\end{itemize}

\subsection{Neuronske mreže}
Neuronske mreže trenutno predstavljaju metod mašinskog učenja sa najširim spektrom primena. Postoje različite vrste neuronskih mreža, koje su često specijalizovane za konkretne primene. 

Ono sto je zajedničko svim neuronskim mrežama jeste njihova struktura. Svaka neuronska mreža se sastoji od određenog broja elementarnih jedinica - neurona. Veštački neuron (eng. Artificial neuron - AN) predstavlja model biološkog neurona. Po analogiji sa biološkim neuronima, veštački neuroni jedni drugima prosleđuju
signale i izračunavaju nove signale na osnovu onih koji su im prosleđeni. Struktura povezanosti neurona
i način na koji oni vrše izračunavanja određuju o kakvoj mreži se
radi (npr. potpuno povezana neuronska mreza, konvolutivna neuronska mreza, rekurentna neuroska mreza, ...).


\subsubsection{Konvolutivne neuroske mreže}
Konvolutivne neuronske mreže imaju ulazni sloj, izlazni sloj, nekoliko skrivenih slojeva i veliki broj parametara, pomoću kojih mogu da nauče komplikovane obrasce.

Slojevi mogu biti \emph{konvolutivni} ili \emph{slojevi agregacije}. 

Neuronska mreža u svakom sloju uči određeni broj filtera. Skup filtera koji deluju nad istim ulazom nazivamo konvolutivnim slojem.

Slojevi agregacije služe za agregaciju informacija u cilju smanjenja količine izračunavanja i broja parametara u mreži.

\end{document}
